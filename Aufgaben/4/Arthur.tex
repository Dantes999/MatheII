\documentclass[12pt,a4paper]{article}

\usepackage{german}      % Deutsche TeX-Eigenheiten
\usepackage{sectsty}
\usepackage{xcolor}
\usepackage{makeidx}
\makeindex            % damit eine Indexdatei angelegt wird

\usepackage{graphicx}

\usepackage{amsmath}  % allgemeine Mathe-Erweiterungen
\usepackage{amssymb}  % Symbole und Schriftarten
\usepackage{amsthm}   % erweiterte Theorem-Umgebungen

\usepackage{mathrsfs}  % gibt den Befehl "\mathscr{}" f�r sch�ne

\begin{document}
Mathematik für Informatiker II - Arthur Kunze
\section*{Aufgabe 16}
$\vec{a} = \left( \begin{array}{c} -19 \\\ 3 \\\ 7 \ \end{array}\right)\in \mathbb{R}^3 \quad \vec{b} = \left( \begin{array}{c} -4 \\\ 3 \\\ 12 \ \end{array}\right)\in \mathbb{R}^3$\
\\
$\vec{v} = \frac{<\vec{a},\vec{b}>}{||\vec{b}||^2}\cdot \vec{b} \rightarrow <\vec{a},\vec{b}> = (-19)\cdot (-4)+3\cdot 3+7\cdot 12 = 169$\\
\\
$||\vec{b}|| =\sqrt{<\vec{b},\vec{b}>}=\sqrt{(-4)^2+3^2+12^2} = \sqrt{169}$\\
\\
$\rightarrow \vec{u}=\frac{169}{\sqrt{169}^2}\cdot \left( \begin{array}{c} -4 \\\ 3 \\\ 12 \ \end{array}\right)= \left( \begin{array}{c} -4 \\\ 3 \\\ 12 \ \end{array}\right)=\vec{b}$\\
\\
$\vec{v}= \vec{a}-\vec{u} =\vec{a} -\vec{b}= \left( \begin{array}{c} -19 \\\ 3 \\\ 7 \ \end{array}\right) - \left( \begin{array}{c} -4 \\\ 3 \\\ 12 \ \end{array}\right)= \left( \begin{array}{c} -23 \\\ 0 \\\ -5 \ \end{array}\right)$\\
\\
\section*{Aufgabe 17}
$u=\{\vec{u}\in \mathbb{R}^4 | \vec{u}\perp \vec{x} \wedge \vec{u}\perp \vec{y}\} \quad \vec{x}=\left( \begin{array}{c} 1 \\\ 0 \\\ 1 \\\ 0 \ \end{array}\right)\quad \vec{y}=\left( \begin{array}{c} 1 \\\ 2 \\\ 0 \\\ 1 \ \end{array}\right)$\\
\\
$\vec{u}\perp \vec{x} =<\vec{u},\vec{x}> = 0 $\\
\\
$\vec{u}\perp \vec{y} =<\vec{u},\vec{y}> = 0 $\\
\\
$u_1+u_3=0$\\
$u_1+2\cdot u_2+u_4=0$\\
\\
$\rightarrow \left( \begin{array}{c} 1\quad 0\quad 1\quad 0\quad|\quad0 \\\ 1\quad 2\quad 0\quad 1\quad|\quad0 \ \end{array}\right) Operation:Z2-Z1$\\
\\
$\rightarrow \left( \begin{array}{c} 1\quad 0\quad 1\quad 0\quad|\quad0 \\\ 0\quad 2-1\quad 1\quad|\quad0 \ \end{array}\right)=\left\lbrace \begin{array}{c}u_1  +u_3  =0 \\\   2\cdot u_2 -u_3+u_4=0 \ \end{array}\right\rbrace$\\
\\
$\rightarrow 2\cdot u_2 = u_3-u_4 $\\
$\rightarrow u_2=\frac{1}{2}\cdot u_3-\frac{1}{2}\cdot u_4$\\
$u_1=-u_3$\\
$u_3=u_3$ und $u_4=u_4$\\
\\
Basis$=\left\lbrace u_3 \cdot \begin{array}{c} -1\\\ \frac{1}{2} \\\ 1 \\\ 0 \ \end{array}+u_4 \cdot  \begin{array}{c} 0\\\ \frac{-1}{2} \\\ 0 \\\ 1 \ \end{array}    \right\rbrace$\\
\\
\section*{Aufgabe 18}
$A=\begin{pmatrix} 2&1&0&2\\2&0&3&0\\3&{-1}&1&{-2}\\2&1&{-2}&0 \end{pmatrix}$\\
\\
$det(A)=\sum \limits_{j=1}^{4}(-1)^{2+j}\cdot a_{2,j}\cdot det(A_{2,j}) $\\
\\
$=(-1)^3\cdot a_{2,1}\cdot det(A_{2,1})+(-1)^4\cdot a_{2,2}\cdot det(A_{2,2})$\\
$+(-1)^5\cdot a_{2,3}\cdot det(A_{2,3})+(-1)^6\cdot a_{2,4}\cdot det(A_{2,4})$\\
\\
$=-2\cdot det(A_{2,1})-3\cdot det(A_{2,3})$\\
\\
$=-2\cdot \begin{pmatrix} 1&0&2\\{-1}&1&{-2}\\1&{-2}&0 \end{pmatrix}-3\cdot \begin{pmatrix} 2&1&2\\3&{-1}&{-2}\\2&1&0 \end{pmatrix}$\\
\\
$=\begin{pmatrix} -2&0&-4\\2&-2&4\\-2&4&0 \end{pmatrix} - \begin{pmatrix} 6&3&6\\9&-3&-6\\6&3&0 \end{pmatrix}=\begin{pmatrix}-8&-3&-10\\-7&1&10\\-8&1&0 \end{pmatrix}$\\
\\
\section*{Aufgabe 19}
$A=\begin{pmatrix} 0&0&0&s\\s&0&s&3s\\1&3s&0&3s\\s&3&0&5s \end{pmatrix}$\\
\newpage
\noindent
a) regulär $\rightarrow$ det(A) $\ne$0\\
\\
\indent$det(A)=\sum \limits_{j=1}^{4}(-1)^{1+j}\cdot a_{1,j}\cdot det(A_{1,j})$\\
\\
$=(-1)^2\cdot 0 \cdot \begin{pmatrix} 0&s&3s\\3s&0&4s\\3&0&5s \end{pmatrix} + (-1)^3\cdot 0 \cdot \begin{pmatrix} s&s&3s\\1&0&4s\\s&0&5s \end{pmatrix}+ (-1)^4\cdot 0 \cdot \begin{pmatrix} s&0&3s\\1&3s&4s\\s&3&5s \end{pmatrix}$\\
$+(-1)^5\cdot s \cdot \begin{pmatrix} s&0&s\\1&3s&0\\s&3&0 \end{pmatrix}$\\
\\
$=-s \cdot \begin{pmatrix} s&0&s\\1&3s&0\\s&3&0 \end{pmatrix}$\\
\\
\indent NR: $det\begin{pmatrix} s&0&s\\1&3s&0\\s&3&0 \end{pmatrix}=s\cdot \begin{pmatrix} 3s&0\\3&0 \end{pmatrix} -0\cdot\begin{pmatrix} 1&0\\s&0 \end{pmatrix}+s\cdot\begin{pmatrix} 1&3s\\s&3 \end{pmatrix}$
\\
\indent $=s\cdot\begin{pmatrix} 3s&0\\3&0 \end{pmatrix}+s\cdot \begin{pmatrix} 1&3s\\s&3 \end{pmatrix}$\\
\\
\indent \indent NR: $det\begin{pmatrix} 3s&0\\3&0 \end{pmatrix}=3s\cdot 0+0\cdot 3 =0$\\
\\
\indent \indent NR: $det\begin{pmatrix} 1&3s\\s&3 \end{pmatrix}=1\cdot 3+3s\cdot s =3+3s^2$\\
\\
\indent $=s\cdot 0+s\cdot (3+3s^2) = s\cdot(3+3s^2)=3s+3s^3$\\
\\
$=-s\cdot(3s+3s^3) = -3s^2-3s^4$\\
\\
Die Matrix A ist regulär für alle Zahlen für die gilt: $s \ne 0, s \in \mathbb{R}$
\\
b) $s=2$\\
$A=\begin{pmatrix} 0&0&0&2\\2&0&2&6\\1&9&0&8\\2&3&0&10 \end{pmatrix}$\\
\\
$A^{-1}=\frac{1}{det(A)}\cdot adj(A)^T$\\
\\
$det(A)=-3\cdot 2^2-3\cdot 2^4=-3\cdot 4-3\cdot 16=-12-48=-60$\\
\\
$adj(A)=\begin{pmatrix} -132&-12&42&30 \\ 0&0&30&0\\-12&8&12&0\\36&-4&-36&0  \end{pmatrix}$\\
\\
$\rightarrow -\frac{1}{60}\cdot \begin{pmatrix} -132&-12&42&30 \\ 0&0&30&0\\-12&8&12&0\\36&-4&-36&0  \end{pmatrix}^T$\\
\\
$=-\frac{1}{60}\cdot \begin{pmatrix} -132&0&-12&36 \\ -12&0&8&-4\\42&30&12&-36\\30&0&0&0  \end{pmatrix}$\\
\\
\section*{Aufgabe 20}
$x_1+2x_2+x_3=0$\\
$tx_1+x_2-3x_3=0$\\
$-x_1+x_2-x_3=1$\\
\\
$A=\begin{pmatrix} 1&2&1\\t&1&-3\\-1&1&-1 \end{pmatrix}$\\
\\
$det(A)=\begin{vmatrix} 1&2&1\\t&1&-3\\-1&1&-1\end{vmatrix}\xrightarrow{Z2+Z1}\begin{vmatrix} 3&2&\not1\\t+1&1&\not-3\\\not0&\not1&\not-1\end{vmatrix}$\\

Entwicklung nach 3. Zeile:
\\
$=(-1)\cdot \begin{vmatrix} 3&2\\t+1&1\end{vmatrix}=-1\cdot (3\cdot 1-2\cdot (t+1))=-3+2(t+1)$\\
\\
$=-3+2t+2$\\
$=2t-1$\\
\\
\newpage
\noindent
$t \ne 0,5 \quad rang(A)=3$\\
$\indent \indent \quad A^{-1}$existiert\\
$\indent \indent \quad \forall \vec{b}\in \mathbb{R}^3 $\\
$\indent \indent \quad $LGS $A\vec{x}=\vec{b}$ eindeutig lösbar\\
\\
$t = 0,5 \quad rang(A)=3$\\
$\indent \indent \quad A^{-1}$existiert nicht\\
$\indent \indent \quad $LGS $A\vec{x}=\vec{b}$ unlösbar oder mehrdeutig lösbar\\
\end{document} 