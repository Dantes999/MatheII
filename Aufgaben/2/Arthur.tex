\documentclass[12pt,a4paper]{article}

\usepackage{german}      % Deutsche TeX-Eigenheiten
\usepackage{sectsty}
\usepackage{xcolor}
\usepackage{makeidx}
\makeindex            % damit eine Indexdatei angelegt wird

\usepackage{graphicx}

\usepackage{amsmath}  % allgemeine Mathe-Erweiterungen
\usepackage{amssymb}  % Symbole und Schriftarten
\usepackage{amsthm}   % erweiterte Theorem-Umgebungen

\usepackage{mathrsfs}  % gibt den Befehl "\mathscr{}" f�r sch�ne

\begin{document}
Mathematik für Informatiker II - Arthur Kunze
\section*{Aufgabe 6}
\subsection*{a)}
$\sum\limits_{k=0}^{\infty}(\frac{4k-1}{7k+3})^k$\\
\\
Mit dem Wurzelkriterium:\\
$\lim\limits_{k\rightarrow\infty}\frac{1}{\sqrt[k]{|a_k|}} \rightarrow a_k=(\frac{4k-1}{7k+3})^k$\\
\\
$\lim\limits_{k\rightarrow\infty}\frac{1}{\sqrt[k]{|(\frac{4k-1}{7k+3})^k|}} = \lim\limits_{k\rightarrow\infty}\frac{1}{|\frac{4k-1}{7k+3}|} = \lim\limits_{k\rightarrow\infty}\frac{7k+3}{4k+1}=$\textcolor{blue}{$\frac{\infty}{\infty}$}\\
\\
1.Regel L'Hostpital:\\
\\
$=\frac{7}{4} > 1 \rightarrow$ \textbf{divergent}\\
\\
\subsection*{b)}
$\sum\limits_{k=0}^{\infty}\frac{(-2)^k}{1+2^{2k}}$ mit $a_k=\frac{(-2)^k}{1+2^{2k}}$\\
$= -(\frac{2k}{1+4^k})$ für $|a_k| = \frac{2k}{1+4^k} \leq \frac{2^k}{4^k}$\\
$\sum\limits_{k=0}^{\infty}\frac{2^k}{4^k} = 2 \rightarrow $divergent
\newpage
\section*{Aufgabe 7}
$\sum\limits_{k=1}^{\infty}(-1)^{k+1}\frac{3}{\sqrt[5]{k^3}}(x-2)^k$ mit $x_0 = 2; a_k=(-1)^{k+1}\frac{3}{\sqrt[5]{k^3}}$\\
\\
Mit dem Wurzelkriterium:\\
$\lim\limits_{k\rightarrow\infty}\frac{1}{\sqrt[k]{|a_k|}} \rightarrow \lim\limits_{k\rightarrow\infty}\frac{1}{\sqrt[k]{(-1)^{k+1}\frac{3}{\sqrt[5]{k^3}}}} = \lim\limits_{k\rightarrow\infty}\frac{1}{\sqrt[k]{\frac{3}{\sqrt[5]{k^3}}}} = \frac{1}{1} = 1 = \rho$\\
\textcolor{blue}{wegen $\lim\limits_{k\rightarrow\infty}\sqrt[k]{x / k} \rightarrow 1$}\\
$= |x-2| \cdot 1$
$= |x-2| < 1 \rightarrow$ \textbf{Konvergenz}\\
Die Potenzreihe konvergiert für alle reellen Zahlen x mit $1<x<3$\\
\section*{Aufgabe 8}
$\sum\limits_{k=1}^{\infty}(2-\frac{1}{k})^k(x+3)^k$ mit $x_0=3; a_k=(2-\frac{1}{k})^k$\\
\\
Mit dem Wurzelkriterium:\\
$\lim\limits_{k\rightarrow\infty}\frac{1}{\sqrt[k]{|a_k|}} \rightarrow \lim\limits_{k\rightarrow\infty}\frac{1}{\sqrt[k]{(2-\frac{1}{k}^k)}}=\lim\limits_{k\rightarrow\infty}\frac{1}{\sqrt[k]{|2-\frac{1}{k}^k|}}=\lim\limits_{k\rightarrow\infty}\frac{1}{(2-\frac{1}{k})}=\frac{1}{2}=\rho$\\
\textcolor{blue}{wegen $\lim\limits_{k\rightarrow\infty}\frac{1}{k}\rightarrow 0$}
\\
\\
$x = -\frac{7}{2}$
\\
$|-\frac{7}{2}+3|=|3-3.5|=|-0.5|=0.5 = \frac{1}{2} \rightarrow $\textbf{Keine Aussage möglich}
\\
\\
$\rightarrow$ Randpunkte gesondert untersuchen:\\
\\
$\sum\limits_{k=1}^{\infty}(2-\frac{1}{k})^k(-\frac{7}{2}+3)^k = \sum\limits_{k=1}^{\infty}(2-\frac{1}{k})^k(-\frac{1}{2})^k$ mit \textcolor{blue}{$\lim\limits_{k\rightarrow \infty}-(\frac{1}{2})^k\rightarrow 1$}\\
\\
$\rightarrow \lim\limits_{k\rightarrow \infty}(2-\frac{1}{k})^k = 1$ wegen \textcolor{blue}{$\lim\limits_{k\rightarrow \infty}\frac{1}{k}=0 $;$\lim\limits_{k\rightarrow \infty}2^k = 1$}
\\\\
$\rightarrow 1 > \frac{1}{2} \rightarrow $\textbf{divergent}
\\
\\
\\
$x = -3$
\\
$|-3+3|=|3-3|=|0|=0 < \frac{1}{2} \rightarrow $\textbf{absolut konvergent}
\newpage
\section*{Aufgabe 9}
$\int x^2e^xdx=G(x)$\\
$f(x) =e^x\rightarrow f'(x)=e^x$\\
$g(x) =x^2\rightarrow g'(x)=2x$\\
\\
Einsetzten für partielle Integration(\textcolor{blue}{$\int f'(x)g(x)dx=f(x)g(x)-\int f(x)g'(x)dx$}):\\
$\int e^xx^2dx=e^xx^2-\int e^x2x dx$\\
\\
zu $\int e^x2xdx$:\\
$f(x) =e^x\rightarrow f'(x)=e^x$\\
$g(x) =2x\rightarrow g'(x)=2$\\
\\
$\rightarrow \int e^x2x dx = e^x2x-\int e^x2dx = e^x2x-2\int e^x dx$\\
$=e^x2x-2e^x+C$\\
Einsetzten in $e^xx^2-\int e^x2x dx$:\\
$= e^xx^2-(e^x2x-2e^x+C)$\\$= e^xx^2-e^x2x+2e^xC$\\$= e^x(x^2-2x+2)+C$\\\underline{$=e^x(x^2-2x+2) = G(x)$}\\
\section*{Aufgabe 10}
$f(x)=\int \frac{cos(ln(x))}{x}dx$ mit $x>0$\\
\\
Substituieren:\\
$t=ln(x),dt=\frac{1}{x}$\\
\\
Einsetzen:\\
$\int \frac{cos(t)}{x}\frac{1}{x}dt=\int cos(t) dt=sin(x)+C$\\
\\
Re-Substituieren:\\
$F(x)=sin(ln(x))+C$
\end{document} 