\documentclass[12pt,a4paper]{article}

\usepackage{german}      % Deutsche TeX-Eigenheiten
%\usepackage{isolatin1}   % Eingabekodierung mit Umlauten...

\usepackage{makeidx}
\makeindex            % damit eine Indexdatei angelegt wird

\usepackage{graphicx}

\usepackage{amsmath}  % allgemeine Mathe-Erweiterungen
\usepackage{amssymb}  % Symbole und Schriftarten
\usepackage{amsthm}   % erweiterte Theorem-Umgebungen

\usepackage{mathrsfs}  % gibt den Befehl "\mathscr{}" für schöne

\usepackage[noframe]{showframe}
\usepackage{framed}
\renewenvironment{shaded}{%
	\def\FrameCommand{\fboxsep=\FrameSep \colorbox{shadecolor}}%
	\MakeFramed{\advance\hsize-\width \FrameRestore\FrameRestore}}%
{\endMakeFramed}
\definecolor{shadecolor}{gray}{0.75}

\newcommand\underrel[2]{\mathrel{\mathop{#2}\limits_{#1}}}

\begin{document}

\section{21. Die Taylorsche Formel}
Funktion = Summe eines Polynoms + Fehlerterm (Restglied) \\
Problem: Funktionen auf Rechner darstellen ( sin,cos,Expo,..)\\
Grundidee:\\
f(x) durch Polynom $P_n(x)$ approximieren (annähern)\\
Polynom :\hspace{5mm}
$P_n(x) = a_0 + a_1 x + a_2 x^2 + ... + a_n x^n,\ a_k\in\mathbb{R},\ x\in I \subseteq\mathbb{R},\ n\in\mathbb{N}$\\
1. Forderung 0 ist im Intervall enthalten\\
Ansatz:\\
Wert von f an der Stelle x (exakt)\\
= Näherung + Rest\\
$\underbrace{f(x)}_{\text{exakt}}=\underbrace{P_n(x)}_{\text{Näherung}}+\underbrace{R_n(x)}_{\text{Rest (Fehler)}},\ x\in I\ 0\in I$\\
\\
Weitere Forderung:\\
an der Stelle 0 soll der Funktionswert und der Wert der k'ten Ableitung von k=0 (0. Ableitung)\\
$\underbrace{f^{(k)}(0)}_{\text{gegeben}}=P_n^{(k)}(0),\ k=0,1,2,...n$\\
k'te Ableitung eines Polynoms:\\
$$
\begin{matrix}
& P_n(x)&=&a_0+a_1 x+a_2 x^2+a_3 x^3+...+a_n x^n\\
k=1\Rightarrow & \left(P_n(x)\right)'&=&1\cdot a_1+2\cdot a_2 x+3\cdot a_3 x^2+...+n\cdot a_n x^{n-1}\\
k=2\Rightarrow & \left(P_n(x)\right)''&=&2\cdot a_1+2\cdot 3\cdot a_2 x+...+n\cdot\left(n-1\right)\cdot a_n x^{n-2}
\end{matrix}
$$
\begin{shaded}
	$\Rightarrow a_k=\frac{1}{k!}\cdot f^{(k)}(0)$
\end{shaded}

\subsubsection{Näherungspolynom}
$P_n(x)=\sum\limits_{k=0}^{n}\frac{1}{k!}f^{(k)}(0)x^k= \sum\limits_{k=0}^{n}\frac{1}{k!}f^{(k)}(0)\left(x-0\right)^k$\\
Stelle 0 geht als Funktionswert ein\\
Wenn $x^k = (x-0)^k$\\
Problem: 0 nicht im Intervall ?\\
Forderung ist gleich, bzw bezieht sich auf $x_0$\\
$= (x-x_0)^k$\\
$P_n(x)=\sum\limits_{k=0}^{n}\frac{1}{k!}f^{(k)}\left(x_0\right)\left(x-x_0\right)^k$

\subsection{Satz von Taylor}
Funktion f soll in einem Intervall, n+1 mal stetig differenzierbar sein.\\
d.h. Ableitungen existieren und sind stetig\\
Formel:
\begin{shaded}
	$$
	\begin{matrix}
	f(x)=\underbrace{f(x_0)+\frac{f'(x_0)}{1!}\left(x-x_0\right)+\frac{f''(x_0)}{2!}\left(x-x_0\right)^2...+\frac{f^{(n)}(x_0)}{n!}\left(x-x_0\right)^n}_{\text{Taylorpolynom n-ter Ordnung (Hauptteil)}}\\
	\\
	+\underbrace{\frac{f^{(n+1)}(\xi)}{\left(n+1\right)!}\left(x-x_0\right)^{n+1}}_{R_n \text{Restglied von Lagrange}}
	\end{matrix}
	$$
\end{shaded}
Entwicklungspunkt $x_0$ = beliebig, aber fest aus Intervall\\
Zwischenstelle $\xi$ liegt zwischen x und $x_0$, kann also kleiner als x oder auch größer sein.\\
\subsubsection{Fehlerabschätzung}
n+1. Ableitung beschänkt im Intervall I.\\
= Für alle x aus I der Betrag der n+1 Ableitung von f an der Stelle x kleiner 0 einer Konstanten ist\\
$f^{(n+1)}$ beschränkt in I, d.h.:
\begin{shaded}
	$\left|f^{(n+1)}(x)\right|\leq M,\ x\in I$
\end{shaded}
$$
\begin{matrix}
\Rightarrow\left|f(x)-P_n(x)\right|=\left|R_n(x)\right| &=& \left|\frac{1}{\left(n+1\right)!}f^{(n+1)}(\xi)\left(x-x_0\right)^{n+1}\right| \\
\\
&=& \frac{1}{\left(n+1\right)!}\left|\left(x-x_09\right)^{n+1}\right|\left|f^{(n+1)}(\xi)\right| \\
\\
&\leq& \frac{1}{\left(n+1\right)!}\left|\left(x-x_09\right)^{n+1}\right| M \\
\end{matrix}
$$
\newpage
Man sieht:\\
1. Je größer das n, dest kleiner wird der Faktor $1\frac{1}{(1-n)!}$\\
auf Deutsch: mit Großerem n wird die approximation besser\\
2. Je weiter das x von $x_0$ weg liegt, desto größer wird der Bertrag $x-x_0$, \\
desto mehr Einfluss hat der Term auf die Genauigkeit\\
\subsubsection{Beispiel 1}
Die Berechnung von Wurzeln - $\sqrt{42}$\\
mit TaylorPolynom 1. Ordnung; Entwicklungspunkt $x_0$ größer 0\\
$f(x)=\sqrt{x}=x^{\frac{1}{2}}\ \Rightarrow f'(x)=\frac{1}{2}x^{-\frac{1}{2}}=\frac{1}{2}\frac{1}{\sqrt{x}}$\\
$\Rightarrow$ Taylorpolynom $\sqrt{x_0}+\frac{1}{2}\frac{1}{\sqrt{x_0}}\left(x-x_0\right)$\\
setzten x ein ($\sqrt{42}$) und bestimmen $x_0 = 36$\\
$\sqrt{x}\approx\sqrt{x_0}+\frac{1}{2}\frac{1}{\sqrt{x_0}}\left(x-x_0\right)\Rightarrow\sqrt{42}\approx\sqrt{36}+\underbrace{\frac{1}{2}\frac{1}{\sqrt{36}}\left(42-36\right)}_{\frac{6}{12}}=6.5$\\
Umstellen für Fehlerabschätzung des Restglieds
\begin{shaded}
	$f(x)=f(x_0)+\frac{f'(x_0)}{1!}\left(x-x_0\right)+\frac{f''(\xi)}{2!}\left(x-x_0\right)^2$\\
	\\
	$\left|f(x)-f(x_0)-\frac{f'(x_0)}{1!}\left(x-x_0\right)\right|=\left|\frac{f''(\xi)}{2!}\left(x-x_0\right)^2\right|$
\end{shaded}
Fehlerabschätzung des Restglieds 
$R_1=\frac{1}{2!}f''(\xi)\left(x-x_0\right)^2$\\
Brauchen 2. Ableitung\\
$f'(x)=\frac{1}{2}x^{-\frac{1}{2}}=\frac{1}{2}\frac{1}{\sqrt{x}}\Rightarrow f''(x)=-\frac{1}{4}x^{-\frac{3}{2}}=-\frac{1}{4}\frac{1}{\sqrt{x^3}}$\\
Dann einsetzten, ergibt:\\
$\Rightarrow R_1=\frac{1}{2!}\cdot\left(-\frac{1}{4}\frac{1}{\sqrt{\xi^3}}\right)\left(x-x_0\right)^2$\\
Umstellen und einsetzten:\\
$=-\frac{1}{8}\frac{1}{\sqrt{\xi^3}}\left(42-36\right)^2,\ \xi$ zwischen x = 42 und $x_0 = 36$\\
Jetzt alles zusammen packen:\\
$\left|\sqrt{42}-\underbrace{\left(f(x_0)+\frac{f'(x_0)}{1!}\left(x-x_0\right)\right)}_{=6.5}\right|=\underbrace{\left|\frac{f''(\xi)}{2!}\left(x-x_0\right)^2\right|}_{R_1}=\left|-\frac{1}{8}\frac{1}{\sqrt{\xi^3}}(6)^2\right|$\\
\newpage
Den Abschnitt mit $\xi$ verkürzen\\
$\left|-\frac{1}{8}\frac{1}{\sqrt{\xi}}(6)^2\right|=\left|-\frac{36}{8}\frac{1}{\underbrace{\sqrt{\xi^3}}_{>0}}\right|=\frac{18}{4}\frac{1}{\sqrt{\xi^3}}$\\
\fbox{Ein Bruch vergrößert sich, wenn der Nenner verkleinert wird.}\\
\\
Der Schlimmste Fall ist, wenn $\xi$ gleich $x_0$, also 36 ist:\\
$\leq\frac{18}{4}\frac{1}{\sqrt{36^3}}=\frac{18}{4}\cdot\frac{1}{216}=\frac{1}{48}=0.02083..\leq 0.021$\\
\\
Zusammenfassung:\\
Fehlerabschätzung  $\left|\sqrt{42}-6.5\right|=\left|\frac{f''(\xi)}{2!}\left(x-x_0\right)^2\right|=\left|-\frac{1}{8}\frac{1}{\sqrt{\xi^3}}(6)^2\right|\leq 0.021$\\
\hspace*{1cm}$\left|\sqrt{42}-6.5\right|\leq 0.021\Rightarrow$\\
\hspace*{1cm}$6.5-0.021\leq\sqrt{42}\leq 6.5+0.021$\\
Fehlerterm $-\frac{36}{8}\frac{1}{\sqrt{\xi^3}}$\\
ist negativ, d.h. der tatsächliche Wert ist kleiner als der berechnete Wert 6.5.\\
Intervall, in dem die Wurzel liegt:
\hspace*{1cm}$\sqrt{42}\in\left[6.5-0.021, 6.5\right]=\left[6.479, 6.5\right]$\\
\subsubsection{Beispiel 2}
$f(x)=\frac{1}{x^2},\ x_0=1,\ x\geq1,\ n=2$\\
1. Schritt: Ableitungen + $x_0$ einsetzten\\
\hspace*{1cm}$f'(x)=\frac{-2}{x^3}\rightarrow f'(1)=-2$\\
\hspace*{1cm}$f''(x)=\frac{6}{x^4}\rightarrow f''(1)=6$\\
\hspace*{1cm}$f'''(x)=-\frac{24}{x^5}$\\
\\
\hspace*{1cm}$f(x)=\frac{1}{x^2}=1+(-2)\frac{\left(x-1\right)}{1!}+6\frac{\left(x-1\right)^2}{2!}+R_2$\\
\\
Nun kommt die Fehlerabschätzung des Restglieds:\\
Erstmal wieder Kürzen:\\
$\underbrace{1+(-2)\frac{\left(x-1\right)}{1!}+6\frac{\left(x-1\right)^2}{2!}}_{1-2\left(x-1\right)+3\left(x-1\right)^2}$\\
\newpage
Mit dem Restglied:\\
$R_2=\frac{1}{3!}\cdot\frac{-24}{\xi^5}\left(x-1\right)^3=-\frac{4}{\xi^5}\left(x-1\right)^3,\ 1\leq\xi\leq x$\\
\\
Und die Abschätzung: eig 1 Einsetzten und schauen, was passiert\\
$\left|R_2\right|=\left|-\frac{4}{\xi^5}\left(x-1\right)^3\right|\underrel{1\leq\xi\leq x}{=}\left|-\frac{4}{\xi^5}\right|\cdot\left|\left(x-1\right)^3\right|\underrel{1\leq\xi\leq x}{\leq}4\cdot\left|\left(x-1\right)^3\right|$\\
\subsubsection{Beispiel 3}
$f(x)=x+\frac{1}{2}x^2+\frac{1}{6}x^3+e^{-x},\ x\in\left[-1, 1\right]$\\
Wie immer erstmal Ableitungen + Einsetzten:\\
\hspace*{1cm}$f(x)=x+\frac{1}{2}x^2+\frac{1}{6}x^3+e^{-x}\Rightarrow f(0)=e^0=1$\\
\hspace*{1cm}$f'(x)=1+x+\frac{1}{2}x^2-e^{-x}\Rightarrow f'(0)=1-1=0$\\
\hspace*{1cm}$f''(x)=1+x+e^{-x}\Rightarrow f''(0)=1+1=2$\\
Für Restglied wird dritte Ableitung benötigt\\
\hspace*{1cm}$f'''(x)=1-e^{-x}$\\
Für e gilt:\\
\hspace*{1cm}$\frac{1}{e}=\frac{1}{e}-1$\\
Jetzt alles in die Formel einsetzten:\\
Taylorpolynom zweiter Ordnung um Entwicklungspunkt $x_0=0$ aufstellen:\\
\hspace*{1cm}$1+x^2$\\
Restglied und Lage von $\xi$ angeben !!\\
$R_2(x)=\frac{1}{3!}\left(1-e^{-\xi}\right)x^3,$
\begin{shaded}
	$\xi$ zwischen $x_0=0$ und $x\in\left[-1,1\right]$
\end{shaded}
Restglied abschätzen\hspace*{1cm}Dreiecksungleichung $\left|a+b\right|\leq\left|a\right|+\left|b\right|$\\
$\left|R_2(x)\right|=\left|\frac{1}{3!}\left(1-e^{-\xi}\right)x^3\right|\leq\frac{1}{6}\left(\left|1\right|+\left|-e^{-\xi}\right|\right)\left|x^3\right|$\\
\\
worst-case-Abschätzung\hspace*{1cm}$=\frac{1}{6}\left(1+e^{-\xi}\right)\left|x^3\right|$
\begin{shaded}
	$\leq\frac{1}{6}\left(1+e\right)\left|x^3\right|\leq 0.6197\left|x^3\right|$
\end{shaded}
\begin{shaded}
	$\xi$ zwischen $x_0=0$ und $x\in\left[-1,1\right]$
\end{shaded}
Worst Case, wäre $\xi$ gleich -1\\

\section{22. Reihen}
\underline{\textbf{Definition 1}}\\
Es sei $a_k$ eine Zahlenfolge. Durch schrittweise Addition der ersten $n$ Glieder erhält man eine Folge $s_n$ mit den Gliedern:\\
$s_1 = a_1, s_2 = a_2, ... ,s_1 = \sum\limits_{k=1}^{n}a_k$ \\
Die Folge $s_n$ nennt man die zur Folge $a_k$ gehörige \underline{unendliche Reihe}.\\
Das $n$-te Glied heißt $n$-te Partialsumme.\\
\underline{\textbf{Beispiel 1}}\\
$a_k = \frac{1}{k}, k = 1,2,3,..$\\
$s_1=1,s_2=2+\frac{1}{2},s_2=1+\frac{1}{2}+\frac{1}{3},...,s_n=\sum\limits_{k=1}^{n}\frac{1}{k}$
\\
\underline{\textbf{Definition 2}}\\
Falls die Folge $s_n$ der Partialsummen keinen Grenzwert
besitzt, nennt man die Reihe divergent.
Die Reihe heißt konvergent, wenn $s_n$ konvergiert. Dann setzt
man\\ 
$s = \lim \limits_{n \to \infty}s_n=\lim \limits_{n \to \infty}\sum\limits_{k=1}^{n}a_k=\sum\limits_{k=1}^{\infty}a_k$
\\
Im Falle der Konvergenz sagt man die Reihe $\sum\limits_{k=1}^{\infty}a_k$ ist konvergent und nennt $s$ den Grenzwert die Summe der unendlichen Reihe.
\\
\underline{\textbf{Beispiel 2}}\\
$a_k=q^k,k=0,1,..$\\
$s_0=a_0=1$\\
$s_1=a_0+a_1=1+q$\\
$s_2=a_0+a_1+a_2$\\
$s_n=a_0+a_1+...+a_n=\sum\limits_{k=0}^{n}q^k$Hinweis:$\sum\limits_{k=0}^{n}q^k=\frac{1-q^{n+1}}{1-q}$\\
$\lim \limits_{n \to \infty}s_n =\lim \limits_{n \to \infty} \frac{1-q^{n+1}}{1-q}=\frac{1-\lim \limits_{n \to \infty}q^{n+1}}{1-q}$
\begin{shaded}
	$\lim \limits_{n \to \infty}s_n =\lim \limits_{n \to \infty}\sum\limits_{k=0}^{n}q^k=\frac{1-\lim \limits_{n \to \infty}q^{n+1}}{1-q}$
\end{shaded}
\newpage
\textbf{Geometrische Reihe} $\sum\limits_{k=0}^{n}q^k =\frac{1}{1-q},|q|<1$
\\
Für \textcolor{red}{$|q|>1$} wächst der Term $q_n+1$ für $n\rightarrow\infty$ betragsmäßig unbeschränkt,\\ so dass \textbf{Divergenz} der Folge $s_n$ und somit der Reihe vorliegt.
\\
\\
Im Fall \textcolor{red}{$q=1$} gilt für die Partialsumme $s_n=n+1$.\\Damit liegt \textbf{Divergenz} der Reihe vor.
\\
\\
Im Fall \textcolor{red}{$|q|<1$} strebt $q_n+1$ gegen den Grenzwert 0 und die Reihe ist \textbf{konvergent}.
\\
\\
Im Fall \textcolor{red}{$q=-1$} wechselt $s_n$ fortlaufend zwischen den Werten 1 und 0, d.h. es liegt \textbf{Divergenz} vor

\end{document} 

