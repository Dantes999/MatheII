\documentclass[12pt,a4paper]{article}

\usepackage{german}      % Deutsche TeX-Eigenheiten
%\usepackage{isolatin1}   % Eingabekodierung mit Umlauten...
\usepackage{xcolor}
\usepackage{makeidx}
\makeindex            % damit eine Indexdatei angelegt wird

\usepackage{graphicx}

\usepackage{amsmath}  % allgemeine Mathe-Erweiterungen
\usepackage{amssymb}  % Symbole und Schriftarten
\usepackage{amsthm}   % erweiterte Theorem-Umgebungen

\usepackage{mathrsfs}  % gibt den Befehl "\mathscr{}" f�r sch�ne

\begin{document}

\section{Seminar 1 - L'Hospital + Taylorsche Formel}
\subsection{L'Hospital}
\includegraphics[width=0.8\textwidth]{Bilder/S1/1.png}\\
Die Erste Untersucht 0/0\\
Die Zweite setzt vorraus, dass g gegen +/- Unendlich geht\\
\subsubsection{1. Aufgabe}
\includegraphics[width=0.8\textwidth]{Bilder/S1/2.png}\\
Werkzeug:\\
\includegraphics[width=0.3\textwidth]{Bilder/S1/3.png}\\
Damit erstmal umformulieren:\\
\includegraphics[width=0.8\textwidth]{Bilder/S1/4.png}\\
Typ 0/0\\
\newpage
Nächstes Werkzeug:\\
\includegraphics[width=0.3\textwidth]{Bilder/S1/5.png}\\
Daraus ergibt sich:\\
\includegraphics[width=0.6\textwidth]{Bilder/S1/6.png} = 0\\
\subsubsection{2. Aufgabe}
\includegraphics[width=0.2\textwidth]{Bilder/S1/7.png}\\
1. Typ bestimmten: Unendlich(hoch)0\\
Nicht für die Regeln 1/2 geeignet\\
Werkzeug:\\
\includegraphics[width=0.3\textwidth]{Bilder/S1/3.png}\\
Daraus folgt:
\includegraphics[width=0.3\textwidth]{Bilder/S1/8.png}\\
Weiter Umstellen mit Multiplikationstheorem:\includegraphics[width=0.3\textwidth]{Bilder/S1/10.png}\\\\
\includegraphics[width=0.5\textwidth]{Bilder/S1/9.png}\\
Weiter Umstellen, da e-fkt stetig:\\
\includegraphics[width=0.5\textwidth]{Bilder/S1/11.png}\\
Typbestimmung... wie verhält sich der logarithmus ?\\
\includegraphics[width=0.2\textwidth]{Bilder/S1/12.png}\\
Typ - Unendlich * 0\\
\newpage
Werkzeug:\\
\includegraphics[width=0.2\textwidth]{Bilder/S1/13.png}\\
Daraus ergibt sich:
\includegraphics[width=0.4\textwidth]{Bilder/S1/14.png}\\
Was ist, wenn x gegen 0 geht ? = -Unendlich/Unendlich\\
Anwendung 2. L'Hospital regel:
\includegraphics[width=0.2\textwidth]{Bilder/S1/15.png}\\
\includegraphics[width=0.2\textwidth]{Bilder/S1/16.png}\\
Jetzt wieder zusammensetzten:
\includegraphics[width=0.5\textwidth]{Bilder/S1/17.png}\\
\subsubsection{3. Aufgabe}
\includegraphics[width=0.3\textwidth]{Bilder/S1/18.png}\\
\includegraphics[width=1\textwidth]{Bilder/S1/19.png}\\
\includegraphics[width=1\textwidth]{Bilder/S1/20.png}\\
\includegraphics[width=1\textwidth]{Bilder/S1/21.png}\\
\subsubsection{4. Aufgabe}
\includegraphics[width=1\textwidth]{Bilder/S1/22.png}\\
\includegraphics[width=0.5\textwidth]{Bilder/S1/23.png}\\
\includegraphics[width=1\textwidth]{Bilder/S1/24.png}\\
\includegraphics[width=1\textwidth]{Bilder/S1/25.png}\\
\subsubsection{5. Aufgabe}
\includegraphics[width=0.8\textwidth]{Bilder/S1/26.png}\\
\includegraphics[width=1\textwidth]{Bilder/S1/27.png}\\
\includegraphics[width=1\textwidth]{Bilder/S1/28.png}\\
\newpage
$WK=\lim\limits_{k\rightarrow\infty}\frac{1}{\sqrt[k]{|a_k|}} a_k=(\frac{4k-1}{7k+3})^k$\\
$\lim\limits_{k\rightarrow\infty}\frac{1}{\sqrt[k]{|(\frac{4k-1}{7k+3})^k|}} = \lim\limits_{k\rightarrow\infty}\frac{1}{|\frac{4k-1}{7k+3}|} = \lim\limits_{k\rightarrow\infty}\frac{7k+3}{4k+1}=$\textcolor{blue}{$\frac{\infty}{\infty}$}\\
1.Regel:$\lim\limits_{k\rightarrow\infty}\frac{7}{4} > 1 \rightarrow$ \textbf{divergent}\\
\\
\\
$\sum\limits_{k=0}^{\infty}\frac{(-2)^k}{1+2^{2k}}=\sum\limits_{k=0}^{\infty}\frac{(-1)+2^k}{1+4^k}$\\
\\
\\
$\sum\limits_{k=0}^{\infty}\frac{1}{\sqrt[k]{(-1)^{k+1}\frac{3}{\sqrt[5]{k^3}}(x-2)^k}}$\\
\\
\\
$\lim\limits_{k\rightarrow\infty}\frac{1}{\sqrt[k]{\frac{3}{\sqrt[5]{k^3}}}}$\\
\\
\\
$\lim\limits_{k\rightarrow\infty}\frac{k\rightarrow 0}{-1 \cdot k\rightarrow 0}$\\
\\
\\
\\
$|f(x)|=\frac{1}{\sqrt{x+x^3}}\leq ?$
\\
\\
\\
$|f(x)|=\frac{1}{\sqrt{x\cdot(1+x^2)}}\leq \frac{1}{\sqrt{x\cdot(x^2)}}$
\end{document} 